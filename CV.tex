\documentclass[11pt]{article}


\setlength{\parindent}{0pt}
\usepackage{xltxtra}
\usepackage{hyperref}
\hypersetup{hidelinks}
\usepackage{url}
\urlstyle{tt}
\usepackage{xcolor}
\definecolor{CVBlue}{RGB}{23,110,191}
\usepackage{calc}
\usepackage{graphicx}
\usepackage{tikz}
\usetikzlibrary{calc}
\usepackage{fontspec}
\usepackage{xeCJK}
\usepackage{enumitem}
\CJKsetecglue{} %% 取消中文与数字之间的间隙


%% 主文档字体设置
\setmainfont[
    Path = fonts/Main/,
    Extension = .otf,
    BoldFont = texgyretermes-bold.otf, % 加粗字体
]{texgyretermes-regular.otf} % 正文字体

% 中文字体设置
\setCJKmainfont[
    Path = fonts/hansans/,
    Extension = .ttf,
    BoldFont = NotoSansSC-Bold.ttf, % 加粗字体
]{NotoSansSC-Regular.otf} % 正文字体


\usepackage{relsize}
\usepackage{xspace}

% 使用 fontawesome(部分图标)
\usepackage{fontawesome} 

% A4纸,上下左右边距
\usepackage[
    a4paper,
    left=1.2cm,
    right=1.2cm,
    top=1.5cm,
    bottom=1cm,
    nohead
]{geometry}

\renewcommand{\baselinestretch}{1.5} % 行间距设为1.5

\usepackage{titlesec}
\usepackage{enumitem}
\setlist{noitemsep} % 取消列表项间的额外间距
%\setlist{nosep} % 取消所有垂直间距
\setlist[itemize]{topsep=0.25em, leftmargin=*}
\setlist[enumerate]{topsep=0.25em, leftmargin=*}

% --- 用于控制【不同项目之间】的垂直距离 ---
\newlength{\interProjectSpacing}
\setlength{\interProjectSpacing}{0.9em} % <--- 在此调整项目之间的距离
\newcommand{\projectsep}{\vspace{\interProjectSpacing}}

% --- 用于控制【项目标题】与下方【项目描述】的距离 ---
\newlength{\intraProjectTitleSep}
\setlength{\intraProjectTitleSep}{0.4em} % <--- 在此调整标题和描述的距离
\newcommand{\titlebreak}{\\[\intraProjectTitleSep]}

% --- 用于控制【项目描述】与下方【要点列表】的距离 ---
\newlength{\intraProjectListTopSep}
\setlength{\intraProjectListTopSep}{0.2em} % <--- 在此调整描述和列表的距离

% =======================================================================


\titleformat{\section}         % 定制 \section 命令 
{\large\bfseries\raggedright} % 将 section 标题设置为大号、粗体且左对齐
{}{0em}                      % 可用于为所有 section 添加前缀(如“章节...”)
{}                           % 可用于在标题前插入代码
[{\color{CVBlue}\titlerule}]  % 在标题后插入一条横线
\titlespacing*{\section}{0cm}{*1.6}{*1.2}



\begin{document}
\pagenumbering{gobble}

%%%% 利用tikz来定位照片
\begin{tikzpicture}[remember picture, overlay] 
    \node[anchor = north east] at ($(current page.north east)+(-2cm,-1.2cm)$) {\includegraphics[height=3cm]{avatar.jpg}};
  \end{tikzpicture}%
  %%%% 利用tikz来定位学校Logo,这里只在第一页显示
  \begin{tikzpicture}[remember picture, overlay] 
    \node[anchor = north west] at ($(current page.north west)+(0.5cm,+1.0cm)$) {\includegraphics[height=6cm]{zju.png}};
  \end{tikzpicture}%
\centerline{\LARGE\bfseries{蒋黄煜}} 

\centerline{\normalsize{\faPhone\ 188-0687-2026 \quad \faEnvelopeO\ \href{mailto:3240104011@zju.edu.cn}{3240104011@zju.edu.cn}}} 

\centerline{\normalsize{\faGithubSquare\ \href{https://github.com/maksymilan}{https://github.com/sleepingjhy}
    
\section{\makebox[\widthof{\faGraduationCap}][c]{\color{CVBlue}\faGraduationCap}\ 教育背景}    
\textbf{浙江大学} \hfill 2024.9 -- 至今\\[0.5em] % 标题和正文间加一点距离
商务大数据分析(信息管理与信息系统-统计学双学位)\quad 大二 
\begin{itemize}[nosep]
    \item 相关课程:数据结构(91/100)、分布式文件系统与数据库技术(94/100)、商业数据分析的Python基础(91/100)、数学分析H(90/100)、C语言程序设计基础及实验(98/100)、无线网络应用(89/100)
\end{itemize}

\section{\makebox[\widthof{\faUsers}][c]{\color{CVBlue}\faUsers}\ 项目经历}

% --- 第一个项目 ---
% 将标题行末尾的 \\ 替换为 \titlebreak 命令
\textbf{由 Deepseek API 驱动的二次元游戏人物桌宠 Agent} \hfill 2026.01 -- 至今 \titlebreak
项目描述:为满足相关调研中同学日益增长的情感需求,我基于 \textbf{PySide6} 独立开发了一个 Windows 桌宠应用,支持\textbf{多实例桌宠、拖拽与状态动画、系统托盘、右键菜单、设置持久化、内置音乐播放器与模拟手机聊天界面},具备 EXE 打包分发能力以满足普通同学的快速上手使用。
% 在 itemize 的选项中,使用 topsep=\intraProjectListTopSep 来控制上边距

\begin{itemize}[nosep, topsep=\intraProjectListTopSep]
    \item \textbf{多实例并发管理与同步}:后端采用 \textbf{Python} 语言,利用 \textbf{QMediaPlayer/QAudioOutput} 和 \textbf{QFileSystemWatcher} 实现\textbf{音乐播放引擎(列表循环/单曲循环/随机、进度与音量控制、文件监听自动同步)},执行多实例管理与全局同步(批量增减实例、策略联动),运用 \textbf{JSON 持久化}实现透明度、图标显示优先级、多开数量、关闭策略等配置的持久化,确保应用长期驻留稳定,配置重启可恢复,用户操作链路完整。
    \item \textbf{调用 API 与语音模型实现有声对话}:运用\textbf{网页搜索 skill} 让大模型了解角色设定,运用 \textbf{ASR} 与 \textbf{TTS} 技术实现了 Agent 与用户的\textbf{实时语音对话}。提高了用户的陪伴感和满足感。
    \item \textbf{可视化 GUI}:前端使用 \textbf{PySide6(QtWidgets/QtGui/QtCore)、QSS、Qt 资源系统(qrc)} 框架构建了\textbf{设置页 + 音乐页 + 关于页 + 聊天页的多页面 UI}。实现右键实例弹出菜单与主界面双向实时同步(音量/播放模式/实例设置),实现动画渲染与镜像移动(GIF 状态机:移动/拖拽/休息)。
\end{itemize}

% 使用 \projectsep 命令来分隔两个项目
\projectsep

\textbf{中国知网大规模文献爬虫} \hfill 2025.09 -- 2026.02 \titlebreak
项目描述:为辅助导师和学长有关中文文献AI率检测科研项目的数据集构建,本项目旨在高效爬取知网上指定期刊的中文文献,我运用 Codex 模型负责构建了 \textbf{IP 访问限制恢复机制}和修复了若干bug。

\begin{itemize}[nosep, topsep=\intraProjectListTopSep]
    \item \textbf{构建 IP 访问限制恢复机制}:设计并实现爬虫的访问限制自动恢复链路:检测到 IP/会话受限后,触发重登录流程(关闭旧会话→重新初始化浏览器→IP 自动登录→Cookie 更新→任务续跑),使得爬虫任务的人工校验次数从每15篇一次下降至0。
    \item \textbf{精确检索中文期刊}:采用知网 \textbf{HTML} 网页代码中期刊名所在位置的具体标签进行二次检索,防止默认检索中出现同名英文期刊导致中文数据集受污染,检索准确率上升至 \textbf{100\%} 。
    \item \textbf{修复遗漏爬取进度bug}:将爬取状态为 partial 的期刊也在初始加载时纳入爬取名单,防止不同年份的数据集规模出现断层。
\end{itemize}

\projectsep

% --- 第三个项目 ---
\textbf{基于 SQL Server 的餐厅订单管理系统} \hfill 2025.09 -- 2025.11 \titlebreak
项目描述:为辅助餐厅管理者管理订单,本项目设计了一个基于 SQL Server 的餐厅订单管理系统,实现了订单数据的自动化更新、管理。
\begin{itemize}[nosep, topsep=\intraProjectListTopSep]
    \item \textbf{数据库系统设计与建模}:我完成了餐厅订单管理系统的全流程数据库设计,包括\textbf{需求分析、概念结构设计(ER 图)、逻辑结构设计(第三范式)}。设计并实现了\textbf{11张核心业务表}(如订单、菜品、食材、员工、分店等),并\textbf{建立主外键关系}和\textbf{索引优化}高频查询。
    \item \textbf{订单计数}:我编写了多个复杂\textbf{存储过程},如\textbf{订单创建、订单支付、采购收货、销售日报生成、会员等级自动升级}等。使用\textbf{事务回滚}保障数据一致性,结合视图简化查询逻辑。
    \item \textbf{权限控制与安全性}:我设计了五类\textbf{数据库角色},并通过\textbf{角色授权}、DENY/REVOKE 控制表级与列级权限。为顾客\textbf{创建个性化视图},使用 \textbf{SESSION_CONTEXT} 实现行级数据隔离,保障薪资、成本等敏感数据安全,满足实际餐饮系统的合规性与职责分离需求。
\end{itemize}

\section{\makebox[\widthof{\faCogs}][c]{\color{CVBlue}\faCogs}\ 技术栈}
\begin{itemize}[nosep]
    \item \textbf{编程语言:} \textbf{Python}, SQL, C
    \item \textbf{AI 开发工具:} \textbf{VSCode、Claude Sonnet 4.5、GPT-5.3-Codex}
    \item \textbf{操作系统:} Windows
\end{itemize}
\section{\makebox[\widthof{\faGraduationCap}][c]{\color{CVBlue}\faList}\ 获奖情况}
\begin{itemize}
    \item 浙江大学2024-2025学年国家励志奖学金 \hfill 2025.10
    \item 浙江大学2024-2025学年三等奖学金 \hfill 2025.10
    \item 浙江大学2024-2025学年学业优秀标兵 \hfill 2025.10
    
\end{itemize}
    
\section{\makebox[\widthof{\faInfo}][c]{\color{CVBlue}\faInfo}\ 其他}
\begin{itemize}[parsep=0.5ex]
    \item \textbf{GitHub:} \href{https://github.com/sleepingjhy}{https://github.com/sleepingjhy} 
    \item \textbf{英语水平:} CET-4 607分, CET-6
\end{itemize}
\end{document}
